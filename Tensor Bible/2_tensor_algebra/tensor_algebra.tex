\chapter{Tensor Algebra}
\label{chapter:Tensor-Algebra}

%&& a &= b & \text{Proof \ref{proof:}} \label{equation:}

aaa

\section{Dyads and Dyadics}

tensor product spaces

order and covector-vector degree

\section{Contractions}

\section{Equivalence to Linear Algebra}

isomorphic to linear algebra if appropriate combination of basis/cobasis

linearity

coordinate vs non coordinate basis

components

deltas and epsilons, symbols, only up to rank 2, etc

einstein notation

tensor as operator

basis independence

isomorphism

definitions contrived to fulfill certain properties

using dyadic symbol and not

defining property of dyadic, compare to linear algebra

simple properties of each product (distributive, associative, commutative) that may depend on the operands

linear algebra comparison

computing in canonical basis

introducing the dot product

introducing the cross product

introducing the dyadic product

specifically takes two vectors

non commutative

scalar rule

simple contractions

double contractions

change of basis

invariants of vectors and tensors

symmetric, skew-symmetric

complex numbers

operations between tensors (to fit with vectors)

\begin{itemize}
	\item $\phi, \kappa \in \mathbb{R}$
	\item $\vb{a}, \vb{b}, \vb{c}, \vb{d} \in \mathbb{R}^{n}$
	\item $\vb{x}, \vb{y} \in \mathbb{R}^{m}$
	\item $\vb*{\omega} \in \mathbb{R}^{p}$
	\item $\vb{u}, \vb{v}, \vb{w} \in \mathbb{R}^{3}$
	\item $\vb{Q}, \vb{R} \in \mathbb{R}^{m \times m}$
	\item $\vb{S}, \vb{T} \in \mathbb{R}^{m \times n}$
	\item $\vb{U} \in \mathbb{R}^{n \times p}$
	\item $\vb{A}, \vb{B} \in \mathbb{R}^{3 \times 3}$
\end{itemize}

INTRODUCING VECTORS

\begin{flalign}
	&& \vb{S} &= \sum_{i = 1}^{m} \sum_{j = 1}^{n} S_{ij} \vb{\hat{e}}_{i} \otimes \vb{\hat{e}}_{j} & \label{equation:tens} \\
	&& \vb{u} \cross \vb{v} &= \sum_{i = 1}^{3} \sum_{j = 1}^{3} \sum_{k = 1}^{3} u_{i} v_{j} \epsilon_{ijk} \vb{\hat{e}}_{k} & \\
\end{flalign}

INTRODUCING TENSORS

\begin{flalign}
	&& \left( \vb{a} \otimes \vb{x} \right) \vb{y} &:= \vb{a} \left( \vb{x} \vdot \vb{y} \right) & \label{equation:vec_dyad_vec_dot_vec} \\
	&& \vb{a} \vdot \left( \vb{b} \otimes \vb{x} \right) &:= \left( \vb{a} \vdot \vb{b} \right) \vb{x} & \label{equation:vec_dot_vec_dyad_vec} \\
	&& \vb{x} \vdot \left( \vb{y} \otimes \vb{a} \right) \vb{b} &= \dots & \text{Proof \ref{proof:vec_dot_vec_dyad_vec_dot_vec}} \label{equation:vec_dot_vec_dyad_vec_dot_vec} \\
	&& \left( \vb{a} \otimes \vb{x} \right)^{T} &:= \vb{x} \otimes \vb{a} & \\
	&& \trace \left( \vb{a} \otimes \vb{b} \right) &:= \vb{a} \vdot \vb{b} & \\
	&& \left( \vb{a} \otimes \vb{x} \right) \left( \vb{y} \otimes \vb*{\omega} \right) &= \left( \vb{x} \vdot \vb{y} \right) \left( \vb{a} \otimes \vb*{\omega} \right) & \text{Proof \ref{proof:vec_dyad_vec_dot_vec_dyad_vec}} \label{equation:vec_dyad_vec_dot_vec_dyad_vec} \\
	&& \left( \vb{a} \otimes \vb{x} \right) \vddot \left( \vb{b} \otimes \vb{y} \right) &:= \left( \vb{a} \vdot \vb{b} \right) \left( \vb{x} \vdot \vb{y} \right) & \label{equation:vec_dyad_vec_double_vec_dyad_vec} \\
	&& \vb{a} \otimes \vb{x} &= \sum_{i = 1}^{n} \sum_{j = 1}^{m} a_{i} x_{j} \vb{\hat{e}}_{i} \otimes \vb{\hat{e}}_{j} & \\
\end{flalign}

CONTRACTIONS

\begin{flalign}
	&& \vb{a} \vdot \vb{b} &= \sum_{i = 1}^{n} a_{i} b_{i} & \text{Proof \ref{proof:vec_dot_vec}} \label{equation:vec_dot_vec} \\
	&& \vb{S} \vb{a} = \vb{S} \vdot \vb{a} &= \sum_{i = 1}^{m} \sum_{j = 1}^{n} S_{ij} a_{j} \vb{\hat{e}}_{i} & \text{Proof \ref{proof:tens_dot_vec}} \label{equation:tens_dot_vec} \\
	&& \vb{x} \vdot \vb{S} &= \sum_{i = 1}^{m} \sum_{k = 1}^{n} x_{i} S_{ik} \vb{\hat{e}}_{k} & \text{Proof \ref{proof:vec_dot_tens}} \label{equation:vec_dot_tens} \\
	&& \vb{S} \vb{U} &= \sum_{i = 1}^{m} \sum_{j = 1}^{n} \sum_{\ell = 1}^{p} S_{ij} U_{j\ell} \left( \vb{\hat{e}}_{i} \otimes \vb{\hat{e}}_{\ell} \right) & \text{Proof \ref{proof:tens_dot_tens}} \label{equation:tens_dot_tens} \\
	&& \vb{S} \vddot \vb{T} &= \sum_{i = 1}^{m} \sum_{j = 1}^{n} S_{ij} T_{ij} & \text{Proof \ref{proof:tens_double_tens}} \label{equation:tens_double_tens}
\end{flalign}

\section{Miscellaneous Operations}
\label{section:miscellaneous_operations}

\subsection{Change of Basis}
\label{section:change_of_basis}

\section{Helpful Identities}
\label{section:helpful_identities}

\begin{flalign}
	&& \vb{S} \vb{a} &= \vb{a} \vdot \left( \vb{S}^{T} \right) & \text{Proof \ref{proof:tens_dot_vec_2}} & \label{equation:tens_dot_vec_2} \\
	&& \left( \vb{x} \vdot \vb{S} \right) \vdot \vb{a} &= \vb{x} \vdot \left( \vb{S} \vb{a} \right) & \text{Proof \ref{proof:vec_dot_tens_dot_vec}} \label{equation:vec_dot_tens_dot_vec} \\
	&& \left( \vb{S}^{T} \right) \vddot \vb{V} &= \vb{S} \vddot \left( \vb{V}^{T} \right) & \text{Proof \ref{proof:tens_trans_double_tens}} & \label{equation:tens_trans_double_tens}
\end{flalign}